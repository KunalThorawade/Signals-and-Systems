\let\negmedspace\undefined
\let\negthickspace\undefined
\documentclass[journal,12pt,twocolumn]{IEEEtran}
\usepackage{cite}
\usepackage{amsmath,amssymb,amsfonts,amsthm}
\usepackage{algorithmic}
\usepackage{graphicx}
\usepackage{textcomp}
\usepackage{xcolor}
\usepackage{txfonts}
\usepackage{listings}
\usepackage{enumitem}
\usepackage{mathtools}
\usepackage{gensymb}
\usepackage[breaklinks=true]{hyperref}
\usepackage{tkz-euclide} % loads  TikZ and tkz-base
\usepackage{listings}
\usepackage{gvv}
\usepackage{circuitikz}

\newtheorem{theorem}{Theorem}[section]
\newtheorem{problem}{Problem}
\newtheorem{proposition}{Proposition}[section]
\newtheorem{lemma}{Lemma}[section]
\newtheorem{corollary}[theorem]{Corollary}
\newtheorem{example}{Example}[section]
\newtheorem{definition}[problem]{Definition}

\newcommand{\BEQA}{\begin{eqnarray}}
\newcommand{\EEQA}{\end{eqnarray}}
\newcommand{\define}{\stackrel{\triangle}{=}}
\theoremstyle{remark}
\newtheorem{rem}{Remark}

\graphicspath{./figs/}

%\bibliographystyle{ieeetr}
\begin{document}
%

\bibliographystyle{IEEEtran}


\vspace{3cm}

\title{
	%	\logo{
	Gate Assignment

	\large{EE:1205 Signals and Systems}

	Indian Institute of Technology, Hyderabad
	%	}
}
\author{Kunal Thorawade

EE23BTECH11035
}	
\maketitle


\newpage

%\tableofcontents

\bigskip
 
 \renewcommand{\thefigure}{\theenumi}
 \renewcommand{\thetable}{\arabic{table}}
 %\renewcommand{\theequation}{\theenumi}

 \textbf{Question}:
 In the circuit shown below, the amplitudes of the voltage across the resistor and the capacitor are equal. What is the value of the angular frequency $\omega_o$ (in rad/s)? 
 (Round off the answer to one decimal place.)
 \hfill(GATE BM 32 2023)
 \begin{circuitikz}
	     % Voltage source
	     \draw (0,0) to[sV, v=$100\cos(\omega_{0} t)$] (0,2);
	         
		     % Resistor
		         \draw (0,2) to[R, l=$1\text{ k}\Omega$] (3,2);
			     
			         % Capacitor
				     \draw (3,2) to[C, l=$100\mu\text{F}$] (3,0);
				         % Ground
					     \draw (3,0) -- (0,0);
 \end{circuitikz}

 \solution 

 \begin{table}[ht]
  \centering
    \begin{tabular}{|c|c|}
        \hline
	   \textbf{ Parameter} & \textbf{Description} \\
	        \hline
		     $\rho_1$ & Density of Liquid \\
		          \hline
			       $\rho$ & Density of cork \\
			            \hline
				         $h$ & Height of cylindrical cork \\
					      \hline
					           $x$ & Displacement \\
						        \hline
							    $T$ & Time period \\
							        \hline
								    $A$ & Base area of cylindrical cork \\
								        \hline
									    $F_R$ & Restoring Force \\
									        \hline
										    $a$ & Acceleration \\
										        \hline
											    $\omega$ & Angular Frequency \\
											        \hline
												    $m = \rho Ah$ & Mass of cylindrical cork \\
												        \hline
													  \end{tabular}
													    \vspace{2mm}
													      \caption{Parameter Table}
													        \label{11.14.18}
														\end{table}

 \begin{align}
	 R &\stackrel{\mathcal{L}}{\longleftrightarrow} R \\
	 C &\stackrel{\mathcal{L}}{\longleftrightarrow} \frac{1}{sC} \\
 \end{align}
 \begin{circuitikz}
	 % Voltage source
	 \draw (0,0) to[sV, v=$V_s(s)$] (0,2);
	 % Resistor
	 \draw (0,2) to[R, l=$R$, i=$I(s)$] (3,2);
	 % Capacitor
	 \draw (3,2) to[C, l=$\frac{1}{sC}$] (3,0);
	 % Ground
	 \draw (3,0) -- (0,0);
 \end{circuitikz}

 \begin{align}
	 H\brak{s} &= \frac{I\brak{s}}{V\brak{s}} \\
	 H\brak{s} &= \frac{1}{R+\frac{1}{sC}} \\
	 H\brak{s} &= \frac{1}{10^3 + \frac{10^4}{s}} \\
	 H\brak{s} &= \frac{10^{-3}s}{10 + s} 
 \end{align}
 Put $s = j\omega$,
 \begin{align}
	 H\brak{\omega} &= \frac{10^{-3}j\omega}{10 + j\omega} \\
	 \abs{H\brak{\omega}} &= \frac{10^{-3}\omega}{\sqrt{100 + \omega^2}}
 \end{align}
 At resonant frequency $H\brak{\omega}$ will have maximum value, $H\brak{\omega}$ will be maximum at $\omega = 10$ \\
 \begin{align}
	     \therefore \omega_0 = 10.0  rad/s
 \end{align}
 \end{document}
