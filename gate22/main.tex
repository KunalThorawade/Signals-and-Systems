\let\negmedspace\undefined
\let\negthickspace\undefined
\documentclass[journal,12pt,twocolumn]{IEEEtran}
\usepackage{cite}
\usepackage{amsmath,amssymb,amsfonts,amsthm}
\usepackage{algorithmic}
\usepackage{graphicx}
\usepackage{textcomp}
\usepackage{xcolor}
\usepackage{txfonts}
\usepackage{listings}
\usepackage{enumitem}
\usepackage{mathtools}
\usepackage{gensymb}
\usepackage[breaklinks=true]{hyperref}
\usepackage{tkz-euclide} % loads  TikZ and tkz-base
\usepackage{listings}
\usepackage{gvv}
\usepackage{circuitikz}

\newtheorem{theorem}{Theorem}[section]
\newtheorem{problem}{Problem}
\newtheorem{proposition}{Proposition}[section]
\newtheorem{lemma}{Lemma}[section]
\newtheorem{corollary}[theorem]{Corollary}
\newtheorem{example}{Example}[section]
\newtheorem{definition}[problem]{Definition}

\newcommand{\BEQA}{\begin{eqnarray}}
\newcommand{\EEQA}{\end{eqnarray}}
\newcommand{\define}{\stackrel{\triangle}{=}}
\theoremstyle{remark}
\newtheorem{rem}{Remark}

\graphicspath{./figs/}

%\bibliographystyle{ieeetr}
\begin{document}
%

\bibliographystyle{IEEEtran}


\vspace{3cm}

\title{
%	\logo{
Gate Assignment

\large{EE:1205 Signals and Systems}

Indian Institute of Technology, Hyderabad
%	}
}
\author{Kunal Thorawade

EE23BTECH11035
}	
\maketitle


\newpage

%\tableofcontents

\bigskip
 
 \renewcommand{\thefigure}{\theenumi}
 \renewcommand{\thetable}{\arabic{table}}
 \renewcommand{\thefigure}{\arabic{figure}}
 %\renewcommand{\theequation}{\theenumi}

 \textbf{Question}:
 A Spectrometer is used to detect plasma oscillations in a sample. The spectrometer 
 can work in the range of $3$ x $10^{12}$ rad s$^{-1}$ to $30$ x $10^{12}$ rad s$^{-1}$. The minimum carrier concentration that can be detected by using this spectrometer is $n$ x $10^{21}$ m$^{-3}$. The value of $n$ is \underline{\hspace{2cm}}. (Round off to two decimal places) \\
 (Charge on electron $= -1.6$ x $10^{-19} $ C$^{-1}$, mass of electron = $9.1$ x $10^{-31}$ kg and $\epsilon_0 = 8.85$ x $10^{-12}$ C$^{2}$ N$^{-1}$ m$^{-2}$ ) \\
 \solution 
 \begin{table}[ht]
	  \centering
	    \begin{tabular}{|c|c|c|}
		        \hline
			   \textbf{ Parameter} & \textbf{Value} & \textbf{Description} \\
			       \hline
			           $v\brak{t}$ & $100cos\brak{\omega_0 t}$ & Input Voltage \\
				       \hline
				           $R$ & $1\text{ k}\Omega$ & Resistance \\
					       \hline
					           $C$ & $100\mu\text{F}$ & Capacitance \\
						       \hline
						           $\omega_0$ & ? & Angular Frequency  \\
							       \hline
							           $Z_R = R$ & $10^3$ & Impedance for resistor  \\
								       \hline
								           $Z_C = \frac{1}{sC}$ & $\frac{10^{4}}{s}$ & Impedance for capacitor  \\
									       \hline
									           $H\brak{s}$ & $\frac{I\brak{s}}{V\brak{s}}$ & Transfer Function \\
										       \hline
										         \end{tabular}
											   \vspace{2mm}
											     \caption{Parameter Table}
											       \label{BM_23_32}
\end{table}

 \begin{align}
     \Delta\omega_p &= \sqrt{\frac{n_0e^2}{m\epsilon_0}} \\
         \implies n_0 &= \frac{\brak{\Delta\omega_p}^2m\epsilon_0}{e^2} \\
	     n_0 &= \frac{\brak{27 \text{x} 10^{12}}^2 \text{x} \brak{9.1 \text{x} 10^{31} }\text{x} \brak{8.85 \text{x} 10^{-12}}}{\brak{-1.6 \text{x} 10^{-19}}^2} \\
	         \therefore n_0 &= 2.83 \text{x} 10^{21} \text{m}^{-3} \\
		     n &= n_0 \text{x} 10^{-21} \\
		         \therefore n &= 2.83
			 \end{align}
			 \end{document}
