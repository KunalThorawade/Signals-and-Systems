\let\negmedspace\undefined
\let\negthickspace\undefined
\documentclass[journal,12pt,twocolumn]{IEEEtran}
\usepackage{cite}
\usepackage{amsmath,amssymb,amsfonts,amsthm}
\usepackage{algorithmic}
\usepackage{graphicx}
\usepackage{textcomp}
\usepackage{xcolor}
\usepackage{txfonts}
\usepackage{listings}
\usepackage{enumitem}
\usepackage{mathtools}
\usepackage{gensymb}
\usepackage[breaklinks=true]{hyperref}
\usepackage{tkz-euclide} % loads  TikZ and tkz-base
\usepackage{listings}
\usepackage{gvv}
\usepackage{circuitikz}


\newtheorem{theorem}{Theorem}[section]
\newtheorem{problem}{Problem}
\newtheorem{proposition}{Proposition}[section]
\newtheorem{lemma}{Lemma}[section]
\newtheorem{corollary}[theorem]{Corollary}
\newtheorem{example}{Example}[section]
\newtheorem{definition}[problem]{Definition}

\newcommand{\BEQA}{\begin{eqnarray}}
\newcommand{\EEQA}{\end{eqnarray}}
\newcommand{\define}{\stackrel{\triangle}{=}}
\theoremstyle{remark}
\newtheorem{rem}{Remark}

\graphicspath{./figs/}

%\bibliographystyle{ieeetr}
\begin{document}
%

\bibliographystyle{IEEEtran}


\vspace{3cm}

\title{
	%	\logo{
	Gate Assignment

	\large{EE:1205 Signals and Systems}

	Indian Institute of Technology, Hyderabad
	%	}
}
\author{Kunal Thorawade

EE23BTECH11035
}	
\maketitle


\newpage

%\tableofcontents

\bigskip
 
 \renewcommand{\thefigure}{\theenumi}
 \renewcommand{\thetable}{\arabic{table}}
 %\renewcommand{\theequation}{\theenumi}

 \textbf{Question}:
 In the circuit shown below, the amplitudes of the voltage across the resistor and the capacitor are equal. What is the value of the angular frequency $\omega_o$ (in rad/s)? 
 (Round off the answer to one decimal place.)
 \hfill(GATE BM 32 2023)
 \begin{circuitikz}
	     % Voltage source
	     \draw (0,0) to[sV, v=$100\cos(\omega_{o} t)$] (0,2);
	         
		     % Resistor
		         \draw (0,2) to[R, l=$1\text{ k}\Omega$] (3,2);
			     
			         % Capacitor
				     \draw (3,2) to[C, l=$100\mu\text{F}$] (3,0);
				         
					     % Ground
					         \draw (3,0) -- (0,0);
 \end{circuitikz}

 \solution 

 \begin{table}[ht]
  \centering
    \begin{tabular}{|c|c|}
        \hline
	   \textbf{ Parameter} & \textbf{Description} \\
	        \hline
		     $\rho_1$ & Density of Liquid \\
		          \hline
			       $\rho$ & Density of cork \\
			            \hline
				         $h$ & Height of cylindrical cork \\
					      \hline
					           $x$ & Displacement \\
						        \hline
							    $T$ & Time period \\
							        \hline
								    $A$ & Base area of cylindrical cork \\
								        \hline
									    $F_R$ & Restoring Force \\
									        \hline
										    $a$ & Acceleration \\
										        \hline
											    $\omega$ & Angular Frequency \\
											        \hline
												    $m = \rho Ah$ & Mass of cylindrical cork \\
												        \hline
													  \end{tabular}
													    \vspace{2mm}
													      \caption{Parameter Table}
													        \label{11.14.18}
														\end{table}

 \begin{align}
	 V_R &= V_C \\
	 \implies \abs{Z_R} &= \abs{Z_C}    \\
	 R &= \left|\frac{1}{j\omega_o C}\right| \\
	 \omega_o &= \frac{1}{RC} \\
	 \omega_o &= \frac{1}{\brak{1\ \text{k}\Omega}\brak{100\ \mu\text{F}}} \\
	 \therefore\omega_o &= 10.0
 \end{align} 
 \end{document}
